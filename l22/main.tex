\documentclass{article}

\usepackage{amsmath}
\usepackage[russian]{babel}

\usepackage[top=1.5cm, bottom=2.5cm, left=2.5cm, right=2.5cm]{geometry}
\usepackage{ragged2e}
\usepackage{fancyhdr}

% remove header line
\renewcommand{\headrulewidth}{0pt}

\setcounter{page}{28}

\pagestyle{fancy}
\fancyhf{}
\fancyfoot[L]{\LARGE\thepage}


\begin{document}
\LARGE
\begin{justify}



для $(6.1)$, так как, если $(6.1)$ записать в виде $y'=-\dfrac{M(x,y)}{N(x,y)}=f(x,y)$,
то, очевидно, в этом случае будет выполняться равенство
\begin{equation}
    f(tx,ty)=t^0f(x,y)=f(x,y) \tag{6.2} \label{eq:first}
\end{equation}

\quad 2) Если в (\ref{eq:first}) принять $t = \dfrac{1}{x}(x \ne 0)$, то (\ref{eq:first})
преобразуется к следующему эквивалентному виду:
\begin{equation*}
    f(tx, ty)=t^0f(x,y)=\left(\dfrac{1}{x}\right)^0f(x,y)=
    f\left(\dfrac{1}{x}x\dfrac{1}{x}y\right)=
    f\left(1,\dfrac{y}{x}\right)=\varphi\left(\dfrac{y}{x}\right)
\end{equation*}
\\
Таким образом, дифференциальное уравнение 1-го порядка, разрешённое относительно производной,
является \textit{одонородным уравнением}, если его можно в \textit{следующем виде}:
$y'=\varphi\left(\dfrac{y}{x}\right)$

\quad Однородное дифференциальное уравнение решается приведением к уравнению с
разделяющимися переменными с помощью \textit{подстановки} (замены переменных):
\begin{equation}
    \begin{cases}
        y=ux,\\
        x=x
    \end{cases}
    \textnormal{или} \quad
    \begin{cases}
        x=uy,\\
        y=y
    \end{cases}
    \tag{6.3} \label{eq:system}
\end{equation}
\\
Действительно, применяя первую из (\ref{eq:system}) подстановку в уравнении $(6.1)$,
полагая, что функции $M(x,y)$ и $N(x, y)$ являются одонородными функциями относительно
аргументов одинаковой степени однородности $m$, получаем $M(x,ux)dx + N(x,ux)(xdu + udx) = 0$
или $x^mM(1,u)dx + x^mN(1, u)(xdx+udx) = 0$ и, сократив на множитель $x^m$, получаем
следующее уравнение с разделяющимися переменными, к которому сводится данное уравнеие:
\begin{equation*}
    [M(1,u) + N(1,u)u]dx + [xN(1,u)]du = 0
\end{equation*}


\newpage
\textbf{\textit{Пример.}} Решить дифференциальное уравнение $(x+y)dx=xdy$.

\textbf{\textit{Решение.}} Так как функция $x+y$ и $x$ являются однородными функциями первой
степени однородности, данное уравнение -- однородное. Применяем для его решения подстановку
$\begin{aligned}
    \begin{cases}
        y=ux,\\
        x=x
    \end{cases}
\end{aligned}$
и в результате получаем $(x+ux)dx=x(xdu+udx)$. Отметив, что $x=0$ является, очевидно, решением
данного уравнения, полагая, что $x\ne0$ и сокращая на $x\ne0$, имеем $dx+udx=xdu+udx, dx=xdu,
du=\dfrac{dx}{x}, u=\ln x+ \ln C, e^u=Cx$. Проведя обратную замену и учитывая решение $x=0$,
получаем окончательный ответ: $e^{\frac{y}{x}}=Cx, x=0$.

\quad Рассмотрим теперь виды дифференциальных уравнений, приводящихся к однородным уравнениям.
Дифференциальное уравнение вида
\begin{equation}
    y'=f\left(\dfrac{a_1x+b_1y+c_1}{a_2x+b_2y+c_2}\right), \tag{6.4}
\end{equation}
где
$\begin{aligned}
    \begin{vmatrix}
        a_1 && b_1 \\
        a_2 && b_2
    \end{vmatrix}
\end{aligned} \ne 0$
и $c_1^2+c_2^2 \ne 0$ (при этих условиях данное уравнение не относится к уже рассмотренным
типам), решается с помощью \textit{подстановки}:
\begin{equation}
    \begin{cases}
        x=u+\alpha; \\
        y=v+\beta.
    \end{cases} \tag{6.5}
\end{equation}
Здесь числа $\alpha$ и $\beta$ -- решения системы
\begin{equation}
    \begin{cases}
        a_1\alpha+b_1\beta+c_1=0; \\
        a_2\alpha+b_2\beta+c_2=0.
    \end{cases} \tag{6.6}
\end{equation}



\end{justify}
\end{document}
